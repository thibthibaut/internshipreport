\documentclass[12pt,a4paper]{article}
\usepackage[utf8]{inputenc}
\usepackage{amsmath}
\usepackage{amsfonts}
\usepackage{amssymb}
\usepackage{wrapfig}
\usepackage{graphicx}
\usepackage{caption}
\usepackage{dirtytalk}
\usepackage[left=1.5cm,right=1.5cm,top=2cm,bottom=2cm]{geometry}

\newcommand{\seas}{School of Engineering and Applied Sciences}
\newcommand{\hseas}{Harvard School of Engineering and Applied Sciences}
%\renewcommand{\rmdefault}{phv} % Arial
%\renewcommand{\sfdefault}{phv} % Arial

\title{{\Huge Internship ING4}\\ Report\\ \vspace{1cm} Harvard University}
\author{Thibaut Vercueil}


\begin{document}
\maketitle
\newpage
\tableofcontents
\newpage
\section{Acknowledgement}

\section{Introduction}

\section{History}

\subsection{Harvard University}
%TODO WHERE ??


\begin{wrapfigure}{R}{0.2\textwidth}
\centering
\includegraphics[width=0.15\textwidth]{harvardlogo.png}
\caption*{Havard seal}
\end{wrapfigure}
Harvard university is the most ancient university of the United States. It's located in Cambridge, Massachusetts, in Boston's suburb. It has been created in 1636 by vote of the Court of the Massachusetts Bay Colony. The school was name \textit{Harvard College} in 1639, in homage to John Harvard, who had left the school livre 779 and his library of some 400 books. John Harvard was the first donor to the school.\\
%MAYBE TALK ABOUT THE OTHER DONORS
During the following decades, Harvard University never ceased to grow up and it's now the richest University in the world with \$36.4 billion of endowment.

%TODO A graph with some figures

Harvard university include several universities, here is a list of the most important ones:\\
\begin{itemize}
  \item Faculty of Arts and Sciences composed by \\Harvard College \\Continuing Education \\Graduate School of Arts and Sciences \\Harvard John A. Paulson School of Engineering and Applied Sciences
\item Business School
\item Kennedy School of Government
\item Law School
\item Medical School
\item Radcliffe Institute
\item School of Education
\item Harvard T.H. Chan School of Public Health
\end{itemize}


\subsection{Harvard School of Engineering and Applied Sciences}
As you saw earlier ``beeing in Harvard'' without precising which university doesn't mean much, and so, during this in internship, I was affiliated to the Harvard School of Engineering and Applied Sciences, later called SEAS — Litte story: the school name changed during my journey, as Mr John A. Pauson did a donnation of \$ 400.000.000 to SEAS, so the school was rename after him.\\

The progenitor of the School of Engineering and Applied Sciences was called \textit{Lawrence Scientific School} and was founded in 1847. It was name for Abbott Lawrence, who donated \$50,000 (an unprecedented sum at the time) to create the institution. The was detached from Harvard College, which means it was independ financiery.
At this time, the School saw a diverse group of thinkers and professionals — astronomers, architects, naturalists, engineers, mathematicians, and even philosophers — pass through its doors.\\
At the end of the 19\textsuperscript{th} century, the school suffered the ``Competition'' from the new born Massachusset Institute of Technology (MIT) — Now one of the greatest engineering school in the world. The Harvard president of the time tried to merge the Harvard Scientific School with the MIT, vainly.\\
In 1901, despite the help of Gordon McKay, the school merged with Harvard College and lost his independance.\\

Later, the Harvard Lawrence Scientific School became \textit{The Division of Applied Science} and in 2007, it was rename as the \textit{Harvard School of Engineering and Applied Science}\\
It's a new start for the the School, Venkatesh Narayanamurti, Dean of Harvard School of Engineering and Applied Sciences at the time declared:\\
\say{\textit{Our transition from a Division to a School is not a departure from history—but in some sense, we are coming full circle. The Lawrence School, our progenitor, will be reborn in a new form appropriate for the 21st century.}}\\

Thus, strictly speaking, SEAS is a young school, only 8 years old, and in full growth. Thanks to the 4 milion dollars given by John Paulson, the school will expend and build laboratories in Allston, the city bordering Cambrigdge, on the other side of the river.\\

In order to realise the importance of Harvard engineering school in the world of sciences, here are a few examples of inventions made here:
\begin{itemize}
\item in 1919, the \textbf{crystal oscillator} came out of the Harvard Engineering School’s Cruft Laboratory, invented by George Washington Pierce
\item in 1938, the \textbf{largest cyclotron of the world} (at the time) was constructed at the Graduate School of Engineering's Gordon McKay Engineering Laboratory.
\item in 1977, Bill gates would have graduated from Harvard but he left the university  to found \textbf{Microsoft}, one of the biggest company in the world.
\item in 2004, \textbf{Facebook} was born in a dorm room of Harvard housing, created by Mark Zuckerberg, it's now the biggest social network ever created
\end{itemize}

\subsection{Mazur group}

\hseas{} is composed by several reseach groups. This summer, I worked with the group of eric mazur, Dean of



\end{document}
